\documentclass[italian,12pt,a4paper]{article}
\usepackage[utf8]{inputenc}
\usepackage[T1]{fontenc}
\usepackage{mathtools}
\usepackage{blkarray, bigstrut} %
\usepackage{babel}
\usepackage{graphicx}
\usepackage{subfig}
\usepackage{hyperref}
\usepackage{tikz}
\usepackage{colortbl}
\usepackage{pgf-pie}
\usepackage{algorithm}
\usepackage{algpseudocode}
\usepackage{algorithmicx}
\usepackage{placeins}
\usepackage{svg}
\usepackage{tabularx}
\title{Università degli studi di Bari facoltà di scienze MM.FF.NN}
\date{} % clear date
\hypersetup{
	colorlinks=true,
	linkcolor=black,
	filecolor=magenta,      
	urlcolor=cyan,
	pdfpagemode=FullScreen,
}
\graphicspath{ {./img/} }
\RequirePackage[subfigure]{tocloft}

\cftsetindents{section}{0em}{2em}
\cftsetindents{subsection}{0em}{2em}

\renewcommand\cfttoctitlefont{\hfill\Large\bfseries}
\renewcommand\cftaftertoctitle{\hfill\mbox{}}

\algrenewcommand\algorithmicrequire{\textbf{Input:}}
\algrenewcommand\algorithmicensure{\textbf{Output:}}

\setcounter{tocdepth}{2}
\begin{document}
	\maketitle
	\thispagestyle{empty}
	\begin{center}
		\huge	\textbf{Progetto Metodi Avanzati di Programmazione} \\
		\vspace{20px}
		\Large \textbf{Phosphorus textual-adventure}
	\end{center}
	
	\begin{center}
		by \\
		\Large \textbf{Vito Proscia mat. 735975} \\
	\end{center}
	\vspace{5px}
	\begin{center}
		Email: \href{mailto:v.proscia3@studenti.uniba.it}{v.proscia3@studenti.uniba.it}
	\end{center}
	\vspace{30px}
	\begin{figure}[hb]
		\centering
		\includegraphics[width=5cm]{image.png}
	\end{figure}
	\vspace{50px}
	\begin{center}
		Repository github: \href{https://github.com/Giut0/Phosphorus}{Phosphorus}
	\end{center}
	
	\vfill
	\begin{center}
		Anno accadenico 2022-2023
	\end{center}
	
	\newpage
	
	\tableofcontents
	
	\newpage
	
	\section{Introduzione}
	\subsection{Trama}
	\subsection{Soluzione gioco}
	\subsection{Organizzazione progetto}
	
	
	\section{Dettagli implementativi}
	\subsection{File}
	\subsection{Database}
	\subsection{Threads}
	\subsection{Api REST}
	\subsection{Java Swing}
	
	\section{Conclusioni}

	\section{Considerazioni finali}
	
	\appendix
	\renewcommand{\thesection}{\Roman{section}}
	\section{Appendice Musica}
	\subsection{Composizione}
	\subsection{Produzione}
	
	
\end{document}