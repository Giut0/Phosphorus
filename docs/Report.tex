\documentclass[italian,12pt,a4paper]{article}
\usepackage[utf8]{inputenc}
\usepackage[T1]{fontenc}
\usepackage{mathtools}
\usepackage{blkarray, bigstrut} %
\usepackage{babel}
\usepackage{graphicx}
\usepackage{subfig}
\usepackage{hyperref}
\usepackage{tikz}
\usepackage{colortbl}
\usepackage{pgf-pie}
\usepackage{algorithm}
\usepackage{algpseudocode}
\usepackage{algorithmicx}
\usepackage{placeins}
\usepackage{svg}
\usepackage{tabularx}
\title{Università degli studi di Bari facoltà di scienze MM.FF.NN}
\date{} % clear date
\hypersetup{
	colorlinks=true,
	linkcolor=black,
	filecolor=magenta,      
	urlcolor=cyan,
	pdfpagemode=FullScreen,
}
\graphicspath{ {./img/} }
\RequirePackage[subfigure]{tocloft}

\cftsetindents{section}{0em}{2em}
\cftsetindents{subsection}{0em}{2em}

\renewcommand\cfttoctitlefont{\hfill\Large\bfseries}
\renewcommand\cftaftertoctitle{\hfill\mbox{}}

\algrenewcommand\algorithmicrequire{\textbf{Input:}}
\algrenewcommand\algorithmicensure{\textbf{Output:}}

\setcounter{tocdepth}{2}
\begin{document}
	\maketitle
	\thispagestyle{empty}
	\begin{center}
		\huge	\textbf{Progetto Metodi Avanzati di Programmazione} \\
		\vspace{20px}
		\Large \textbf{Phosphorus textual-adventure}
	\end{center}
	
	\begin{center}
		by \\
		\Large \textbf{Vito Proscia mat. 735975} \\
	\end{center}
	\vspace{5px}
	\begin{center}
		Email: \href{mailto:v.proscia3@studenti.uniba.it}{v.proscia3@studenti.uniba.it}
	\end{center}
	\vspace{30px}
	\begin{figure}[hb]
		\centering
		\includegraphics[width=5cm]{image.png}
	\end{figure}
	\vspace{50px}
	\begin{center}
		Repository github: \href{https://github.com/Giut0/Phosphorus}{Phosphorus}
	\end{center}
	
	\vfill
	\begin{center}
		Anno accadenico 2022-2023
	\end{center}
	
	\newpage
	
	\tableofcontents
	
	\newpage
	
	\section{Introduzione}
	\subsection{Definizione obiettivo principale}
	L'obiettivo principale del progetto è la realizzazione di un'avventura testuale in Java, inglobando, in essa, le tecniche e gli argomenti trattati durante il corso di Metodi Avanzati di Programmazione.
	\subsection{Trama}
		Il protagonista, l’agente f24, si trova su una navicella spaziale in ritorno alla Terra da una missione che ha consistito nel catturare alieni per produrre il fosforo necessario alla sopravvivenza della Terra, infatti, sulla quest'ultima, il fosforo, che riveste un ruolo fondamentale per la sopravvivenza dei vegetali e quindi per il sostentamento dell’uomo è cominciato a diminuire drasticamente, per questo si organizzano spedizioni per catturare alieni in grado di produrlo. \\
		\linebreak
		Inizialmente, f24 si sveglierà dal sonno criogenico nel dormitorio con un ordine, impartito dal comandante, di indagare sulla misteriosa scomparsa di due alieni prigionieri. Il protagonista cercherà i due fuggitivi, districandosi tra le stanze dell’astronave ed interrogando i membri dell’equipaggio, fino a scoprire cosa viene fatto agli alieni prigionieri. Sarà solo a lui decidere se mantenere lo status quo o cambiare la situazione.
	\subsection{Soluzione gioco}
	
		La stanza iniziale è il dormitorio, la prima cosa che si deve fare è prendere la bombola d'ossigeno che sarà utile più avanti, una volta parlato con l'agante a13 bisogna recarsi alla sala meeting, a nord del dormitorio, dopo aver ascoltato il comandante si dovrà prendere la pistola e ci si recherà ad ovest per la mensa, lì ci saranno due scienziati s99 e s00, per proseguire ci si dovrà spostare ad est verso la sala macchine, una volta dentro ci si imbatterà nel primo nemico, l'alieno Orionix, una volta sconfitto bisogneà prendere la chiave dello sgabuzzino, dopo di chè ci si dovrà entrare andando a sud, una voltà all'interno si dovrà prendere il bigliettio che conterrà il codice per accedere al laboratorio (4815) che si troverà appena a nord della sala meeting, una volta entrati troveremo l'ultimo nemico Nebulor, che sarà protetto da un altro scienzito, s00, una volta parlato con quest'ultimo starà al giocatore decidere se eliminare l'ultimo nemico o prendere la modifica delle pistola per cercare di cambiare la situazione.
	
	\subsection{Organizzazione progetto}
	
	
	\section{Dettagli implementativi}
	\subsection{File}
	\subsection{Database}
	\subsection{Threads}
	\subsection{Api REST}
	\subsection{Java Swing}
	
	\section{Conclusioni}

	\section{Considerazioni finali}
	
	\appendix
	\renewcommand{\thesection}{\Roman{section}}
	\section{Appendice Musica}
	\subsection{Composizione}
	\subsection{Produzione}
	
	
\end{document}